%&preformat-disser
\RequirePackage[l2tabu,orthodox]{nag} % Раскомментировав, можно в логе получать рекомендации относительно правильного использования пакетов и предупреждения об устаревших и нерекомендуемых пакетах
% Формат А4, 14pt (ГОСТ Р 7.0.11-2011, 5.3.6)
\documentclass[a4paper,14pt,oneside,openany]{memoir}

\input{include/common/setup}            % общие настройки шаблона
\input{include/common/packages}         % Пакеты общие для диссертации и автореферата
\synopsisfalse                      % Этот документ --- не автореферат
\input{include/dissertation/dispackages}    % Пакеты для диссертации
\input{include/dissertation/userpackages}   % Пакеты для специфических пользовательских задач

\input{include/dissertation/setup}      % Упрощённые настройки шаблона

\input{include/common/newnames}         % Новые переменные, для всего проекта

\input{include/common/data}             % Основные сведения
\input{include/common/fonts}            % Определение шрифтов (частичное)
\input{include/common/styles}           % Стили общие для диссертации и автореферата
\input{include/dissertation/disstyles}  % Стили для диссертации
\input{include/dissertation/userstyles} % Стили для специфических пользовательских задач

%%% Библиография. Выбор движка для реализации %%%
% Здесь только проверка установленного ключа. Сама настройка выбора движка
% размещена в common/setup.tex
\ifnumequal{\value{bibliosel}}{0}{%
    \input{biblio/predefined}   % Встроенная реализация с загрузкой файла через движок bibtex8
}{
    \input{biblio/biblatex}     % Реализация пакетом biblatex через движок biber
}

% Вывести информацию о выбранных опциях в лог сборки
\typeout{Selected options:}
\typeout{Draft mode: \arabic{draft}}
\typeout{Font: \arabic{fontfamily}}
\typeout{AltFont: \arabic{usealtfont}}
\typeout{Bibliography backend: \arabic{bibliosel}}
\typeout{Precompile images: \arabic{imgprecompile}}
% Вывести информацию о версиях используемых библиотек в лог сборки
\listfiles

%%% Управление компиляцией отдельных частей диссертации %%%
% Необходимо сначала иметь полностью скомпилированный документ, чтобы все
% промежуточные файлы были в наличии
% Затем, для вывода отдельных частей можно воспользоваться командой \includeonly
% Ниже примеры использования команды:
%
%\includeonly{Dissertation/part2}
%\includeonly{Dissertation/contents,Dissertation/appendix,Dissertation/conclusion}
%
% Если все команды закомментированы, то документ будет выведен в PDF файл полностью

\begin{document}
%%% Переопределение именований типовых разделов
% https://tex.stackexchange.com/a/156050
\gappto\captionsrussian{\input{include/common/renames}\unskip} % for polyglossia and babel
\input{include/common/renames}

%%% Выпускная квалификационная работа
%%% Структура выпускной квалификационной работы
% Титульный лист pdf

\includepdf[pages=-]{pages/title.pdf}                                                   % Титульный лист pdf
% Задание pdf

%\addcontentsline{toc}{chapter}{ЗАДАНИЕ}    % Добавляем его в оглавление

                                            % Нумерация страниц. Реферат - третья страница. Просто задание занимает 2 страницы
\pagenumbering{gobble}                      % Сбрасываем нумерацию страниц
\pagenumbering{arabic}                      % Включаем нумерацию страниц

\includepdf[pages=-]{pages/task_1.pdf}

\includepdf[pages=-]{pages/task_2.pdf}                                                    % Задание лист pdf
\chapter*{\centering РЕФЕРАТ}
\addcontentsline{toc}{chapter}{РЕФЕРАТ}    % Добавляем его в оглавление

В современном мире программное обеспечение...

Цель работы - разработка программного обеспечения.

Объект исследования - программное обеспечение.

%Методы исследования: аналитический, сравнения инструментов.

Результаты работы: что-то сделано.

Ключевые слова: ключевое слово 1, ключевое слово 2.

Объём и структура работы.
Выпускная квалификационная работа состоит из~введения,
\formbytotal{totalchapter}{глав}{ы}{}{} и
заключения.
Полный объём выпускной квалификационной работы составляет
\formbytotal{TotPages}{страниц}{у}{ы}{}, включая
\formbytotal{totalcount@figure}{рисун}{ок}{ка}{ков}.
Список литературы содержит
\formbytotal{citenum}{наименован}{ие}{ия}{ий}.

\textbf
{
    Из-за задания, количество страниц не совпадает с formbytotal\{TotPages\} на одну.
    formbytotal\{TotPages\} указывает общее количество страниц, а нужно указать последнюю нумерованную страницу.
    Нужно просто руками вычислить (formbytotal\{TotPages\} - 1).
}                                                    % Реферат
\newpage
% Оглавление (ГОСТ Р 7.0.11-2011, 5.2)
%\addcontentsline{toc}{chapter}{ОГЛАВЛЕНИЕ}    % Добавляем его в оглавление
\ifdefmacro{\microtypesetup}{\microtypesetup{protrusion=false}}{} % не рекомендуется применять пакет микротипографики к автоматически генерируемому оглавлению
\tableofcontents*
\addtocontents{toc}{\protect\tocheader}
\endTOCtrue
\ifdefmacro{\microtypesetup}{\microtypesetup{protrusion=true}}{}                                                    % Содержание (Оглавление)
\ifnumequal{\value{contnumfig}}{1}{}{\counterwithout{figure}{chapter}}
\ifnumequal{\value{contnumtab}}{1}{}{\counterwithout{table}{chapter}}
\chapter*{\centering Словарь терминов}                       % Заголовок
\addcontentsline{toc}{chapter}{Словарь терминов}  % Добавляем его в оглавление

\textbf{YAML} (Yet Another Markup Language) - формат сериализации данных.

\textbf{Драйвер} - программа для управления внешним устройством компьютера.

\textbf{Кроссплатформенность} - способность программного обеспечения работать с несколькими операционными системами или аппаратными платформами.

\textbf{Сериализация} - процесс перевода структуры данных в последовательность байтов.

\textbf{Фреймворк} - программный продукт для упрощения поддержки и создания нагруженных или технически сложных проектов.
                                                  % Перечень условных обозначений, символов, сокращений, единиц и терминов
\chapter*{\centering ВВЕДЕНИЕ}
\addcontentsline{toc}{chapter}{ВВЕДЕНИЕ}    % Добавляем его в оглавление

В современном мире...

Актуальность данной выпускной квалификационной работы обусловлена...

Целью данной выпускной квалификационной работы является...

Для достижения поставленной цели необходимо было решить следующие задачи:
\begin{diplomaitemize}
    \item задача 1;
    \item задача 2;
    \item задача 3.
\end{diplomaitemize}

Вода...                                                % Введение
\ifnumequal{\value{contnumfig}}{1}{\counterwithout{figure}{chapter}
}{\counterwithin{figure}{chapter}}
\ifnumequal{\value{contnumtab}}{1}{\counterwithout{table}{chapter}
}{\counterwithin{table}{chapter}}
%%% Глава 1
\chapter{Обзор возможностей системы}\label{ch:chapter_1}

Рассмотрим возможности подготовленного шаблона \LaTeX. % Введение
\section{Форматирование текста}\label{sec:chapter_1/section_1}

Мы можем сделать \textbf{жирный текст} и \textit{курсив}.

\section{Ссылки}\label{sec:chapter_1/section_2}

Сошлёмся на библиографию.
Одна ссылка: \cite{kuznetsov}.
Две ссылки: \cite{kuznetsov, creating_a_ros_package}.
Ещё ссылка: \cite{gost_9294_93}.

\section{Формулы}\label{sec:chapter_1/section_3}

Благодаря пакету \textit{icomma}, \LaTeX одинаково хорошо воспринимает в качестве десятичного разделителя и запятую (\(3,1415\)), и точку (\(3.1415\)).

\subsection{Ненумерованные одиночные формулы}\label{subsec:chapter_1/section_3/subsection_1}

Вот так может выглядеть формула, которую необходимо вставить в строку по тексту: \(x \approx \sin x\) при \(x \to 0\).

\subsection{Другая формула}\label{subsec:chapter_1/section_3/subsection_2}

Другой формулы не будет. Я не знаю математику в \LaTeX.

\subsection{Ещё формула}\label{subsec:chapter_1/section_3/subsection_3}

Ещё формулы не будет. Я не знаю математику в \LaTeX.

\section{Цитата}\label{sec:chapter_1/section_4}

Цитата: <<Lorem ipsum dolor sit amet>>.

\section{Перечисления}\label{sec:chapter_1/section_5}

Я добавил два новых стиля перечислений: diplomaenumerate и diplomaitemize.

\subsection{Числовое перечисление}\label{subsec:chapter_1/section_5/subsection_1}

Числовое перечисление:
\begin{diplomaenumerate}
    \item Первый пункт.
    \item Второй пункт.
    \item Третий пункт.
    \item Четвёртый пункт.
\end{diplomaenumerate}

\subsection{Перечисление пунктами}\label{subsec:chapter_1/section_5/subsection_2}

Перечисление пунктами:
\begin{diplomaitemize}
    \item первый пункт;
    \item второй пункт;
    \item третий пункт;
    \item четвёртый пункт.
\end{diplomaitemize}

\section{Тире и дефисы}\label{sec:chapter_1/section_6}

Стандартное \LaTeX тире --- это было тире.

Ещё есть короткое тире -- это было короткое тире.

Но нормоконтроль требует минус как тире - минус как тире.

\textbf{Тут подходит только последний вариант. Через такое - тире.}    % Раздел 1                                         % Глава 1
%%% Глава 2
\chapter{Пример оформления}\label{ch:chapter_2}

Пример оформления. % Введение
\section{Рисунок и тестовый текст}\label{sec:chapter_2/section_1}

\subsection{Рисунок}\label{subsec:chapter_2/section_1/subsection_1}

На рисунке~\cref{fig:cmake} показан логотип CMake.

\begin{figure}[ht]
    \centerfloat
    {
        \includegraphics[scale=1]{cmake}
    }
    \caption{Логотип CMake}\label{fig:cmake}
\end{figure}

\subsection{Тестовый текст}\label{subsec:chapter_2/section_1/subsection_2}

Lorem ipsum dolor sit amet, consectetur adipiscing elit. Nunc nec fermentum lorem. Nunc vel nulla massa. Donec at cursus ipsum, eu ultricies leo. Etiam malesuada dictum mattis. Aliquam suscipit porta nibh et placerat. Lorem ipsum dolor sit amet, consectetur adipiscing elit. Mauris luctus pretium ex nec elementum. In cursus ornare ipsum, eu aliquet nisl sollicitudin vel. Maecenas placerat eros enim, et pharetra nisi auctor at.

\subsection{Много текста разбитого на абзацы}\label{subsec:chapter_2/section_1/subsection_3}

In interdum neque at lorem aliquet suscipit. In hac habitasse platea dictumst. In enim eros, rutrum eget orci eget, condimentum tempus risus. Cras nec enim erat. Class aptent taciti sociosqu ad litora torquent per conubia nostra, per inceptos himenaeos. Maecenas et felis leo. Vestibulum lacinia metus at libero posuere, et sagittis lectus tincidunt. Pellentesque quis maximus arcu. Vivamus malesuada tellus id lacinia cursus. Donec et odio vel sapien volutpat semper ut vitae nibh. Proin elementum rutrum fringilla. Etiam hendrerit eget massa nec eleifend. Mauris posuere nisi lectus, vel pulvinar libero pulvinar eget. Quisque vehicula lacus et finibus tincidunt.

Sed tincidunt metus ipsum, vitae sodales arcu sodales nec. Quisque vehicula ut metus sed consequat. Pellentesque habitant morbi tristique senectus et netus et malesuada fames ac turpis egestas. Mauris a rhoncus odio. Sed fermentum nisl non lorem faucibus, sed eleifend ipsum tempus. Vestibulum ante ipsum primis in faucibus orci luctus et ultrices posuere cubilia curae; Nullam id ultricies libero. Nam tempus tincidunt arcu, eget maximus felis tempus ac. Etiam ultricies non quam sit amet tempor. Etiam placerat lorem eget nisl efficitur eleifend. Phasellus a odio in quam dictum egestas sed sit amet ligula. Curabitur tincidunt lacus quis eleifend aliquam. Morbi nec nulla dui.

Maecenas ultrices sapien id magna tristique, ut dapibus eros pharetra. Lorem ipsum dolor sit amet, consectetur adipiscing elit. Donec vel diam convallis magna euismod ullamcorper ut eu massa. Cras sit amet libero sit amet lorem eleifend condimentum nec id ex. Fusce aliquet porta neque. Vivamus sodales pretium nisl, ut semper mauris rutrum sit amet. Nunc rhoncus id enim eu porta.

Fusce in arcu sit amet mauris consequat dictum. Pellentesque cursus ultrices commodo. Aenean eu rhoncus ante. Mauris hendrerit sollicitudin tincidunt. Sed quis laoreet tellus. Duis ultricies orci non enim sodales facilisis. Nullam sodales eros eu quam blandit tincidunt. Nulla elementum sodales ornare. Curabitur aliquet enim id velit efficitur maximus. Donec tincidunt, est id placerat varius, massa urna suscipit augue, ac sollicitudin leo elit et massa. Cras et dapibus tellus.

\subsection{Много текста не разбитого на абзацы}\label{subsec:chapter_2/section_1/subsection_4}

Lorem ipsum dolor sit amet, consectetur adipiscing elit. Nunc nec fermentum lorem. Nunc vel nulla massa. Donec at cursus ipsum, eu ultricies leo. Etiam malesuada dictum mattis. Aliquam suscipit porta nibh et placerat. Lorem ipsum dolor sit amet, consectetur adipiscing elit. Mauris luctus pretium ex nec elementum. In cursus ornare ipsum, eu aliquet nisl sollicitudin vel. Maecenas placerat eros enim, et pharetra nisi auctor at.
In interdum neque at lorem aliquet suscipit. In hac habitasse platea dictumst. In enim eros, rutrum eget orci eget, condimentum tempus risus. Cras nec enim erat. Class aptent taciti sociosqu ad litora torquent per conubia nostra, per inceptos himenaeos. Maecenas et felis leo. Vestibulum lacinia metus at libero posuere, et sagittis lectus tincidunt. Pellentesque quis maximus arcu. Vivamus malesuada tellus id lacinia cursus. Donec et odio vel sapien volutpat semper ut vitae nibh. Proin elementum rutrum fringilla. Etiam hendrerit eget massa nec eleifend. Mauris posuere nisi lectus, vel pulvinar libero pulvinar eget. Quisque vehicula lacus et finibus tincidunt.
Sed tincidunt metus ipsum, vitae sodales arcu sodales nec. Quisque vehicula ut metus sed consequat. Pellentesque habitant morbi tristique senectus et netus et malesuada fames ac turpis egestas. Mauris a rhoncus odio. Sed fermentum nisl non lorem faucibus, sed eleifend ipsum tempus. Vestibulum ante ipsum primis in faucibus orci luctus et ultrices posuere cubilia curae; Nullam id ultricies libero. Nam tempus tincidunt arcu, eget maximus felis tempus ac. Etiam ultricies non quam sit amet tempor. Etiam placerat lorem eget nisl efficitur eleifend. Phasellus a odio in quam dictum egestas sed sit amet ligula. Curabitur tincidunt lacus quis eleifend aliquam. Morbi nec nulla dui.
Maecenas ultrices sapien id magna tristique, ut dapibus eros pharetra. Lorem ipsum dolor sit amet, consectetur adipiscing elit. Donec vel diam convallis magna euismod ullamcorper ut eu massa. Cras sit amet libero sit amet lorem eleifend condimentum nec id ex. Fusce aliquet porta neque. Vivamus sodales pretium nisl, ut semper mauris rutrum sit amet. Nunc rhoncus id enim eu porta.
Fusce in arcu sit amet mauris consequat dictum. Pellentesque cursus ultrices commodo. Aenean eu rhoncus ante. Mauris hendrerit sollicitudin tincidunt. Sed quis laoreet tellus. Duis ultricies orci non enim sodales facilisis. Nullam sodales eros eu quam blandit tincidunt. Nulla elementum sodales ornare. Curabitur aliquet enim id velit efficitur maximus. Donec tincidunt, est id placerat varius, massa urna suscipit augue, ac sollicitudin leo elit et massa. Cras et dapibus tellus.
Lorem ipsum dolor sit amet, consectetur adipiscing elit. Nunc nec fermentum lorem. Nunc vel nulla massa. Donec at cursus ipsum, eu ultricies leo. Etiam malesuada dictum mattis. Aliquam suscipit porta nibh et placerat. Lorem ipsum dolor sit amet, consectetur adipiscing elit. Mauris luctus pretium ex nec elementum. In cursus ornare ipsum, eu aliquet nisl sollicitudin vel. Maecenas placerat eros enim, et pharetra nisi auctor at.
In interdum neque at lorem aliquet suscipit. In hac habitasse platea dictumst. In enim eros, rutrum eget orci eget, condimentum tempus risus. Cras nec enim erat. Class aptent taciti sociosqu ad litora torquent per conubia nostra, per inceptos himenaeos. Maecenas et felis leo. Vestibulum lacinia metus at libero posuere, et sagittis lectus tincidunt. Pellentesque quis maximus arcu. Vivamus malesuada tellus id lacinia cursus. Donec et odio vel sapien volutpat semper ut vitae nibh. Proin elementum rutrum fringilla. Etiam hendrerit eget massa nec eleifend. Mauris posuere nisi lectus, vel pulvinar libero pulvinar eget. Quisque vehicula lacus et finibus tincidunt.
Sed tincidunt metus ipsum, vitae sodales arcu sodales nec. Quisque vehicula ut metus sed consequat. Pellentesque habitant morbi tristique senectus et netus et malesuada fames ac turpis egestas. Mauris a rhoncus odio. Sed fermentum nisl non lorem faucibus, sed eleifend ipsum tempus. Vestibulum ante ipsum primis in faucibus orci luctus et ultrices posuere cubilia curae; Nullam id ultricies libero. Nam tempus tincidunt arcu, eget maximus felis tempus ac. Etiam ultricies non quam sit amet tempor. Etiam placerat lorem eget nisl efficitur eleifend. Phasellus a odio in quam dictum egestas sed sit amet ligula. Curabitur tincidunt lacus quis eleifend aliquam. Morbi nec nulla dui.
Maecenas ultrices sapien id magna tristique, ut dapibus eros pharetra. Lorem ipsum dolor sit amet, consectetur adipiscing elit. Donec vel diam convallis magna euismod ullamcorper ut eu massa. Cras sit amet libero sit amet lorem eleifend condimentum nec id ex. Fusce aliquet porta neque. Vivamus sodales pretium nisl, ut semper mauris rutrum sit amet. Nunc rhoncus id enim eu porta.
Fusce in arcu sit amet mauris consequat dictum. Pellentesque cursus ultrices commodo. Aenean eu rhoncus ante. Mauris hendrerit sollicitudin tincidunt. Sed quis laoreet tellus. Duis ultricies orci non enim sodales facilisis. Nullam sodales eros eu quam blandit tincidunt. Nulla elementum sodales ornare. Curabitur aliquet enim id velit efficitur maximus. Donec tincidunt, est id placerat varius, massa urna suscipit augue, ac sollicitudin leo elit et massa. Cras et dapibus tellus.

\subsection{Много текста разбитого на много абзацев}\label{subsec:chapter_2/section_1/subsection_5}

Lorem ipsum dolor sit amet, consectetur adipiscing elit.

Nunc vel euismod velit. Sed vehicula pretium leo, vitae scelerisque diam semper a.

Integer a eros pharetra, egestas massa a, volutpat urna. Morbi dapibus, quam sed ullamcorper scelerisque, purus leo interdum enim, sed venenatis nunc mi vitae turpis.

Pellentesque feugiat accumsan consequat. Vivamus interdum a nulla tempor ultrices.

Suspendisse varius erat a orci faucibus, vitae volutpat quam sollicitudin. Fusce at ipsum lorem.

Morbi efficitur non lectus non pretium. Nunc luctus, magna quis volutpat lacinia, ipsum metus semper nisi, quis feugiat erat ante in felis.

Vestibulum varius semper purus ut eleifend. Vivamus velit diam, dapibus in ultricies ut, aliquet nec ex.

Vivamus sapien elit, condimentum eget mi eu, porttitor finibus sem. Pellentesque quis scelerisque metus. Integer consectetur elementum libero eu mattis. Sed a ultrices ante. Vestibulum efficitur dignissim lacus non accumsan.

Sed a eros quis mi elementum molestie. Proin eget interdum magna. Aliquam mollis lectus et ligula dignissim tempor.

Nam id ipsum consectetur, sollicitudin nunc quis, viverra lorem. Vestibulum id vestibulum ante, ac dignissim leo. Morbi placerat mauris quis felis suscipit cursus. Proin nec lacinia enim, nec aliquam dolor. Nulla vitae metus pulvinar nisi porta bibendum.

Fusce dapibus justo accumsan mollis tincidunt.

In mollis, ligula sit amet pellentesque dapibus, arcu neque ornare diam, et tempor mi est ac sem. Nulla malesuada eu arcu ut rutrum. Vivamus eros metus, imperdiet at egestas nec, rutrum vel arcu. Sed tempor sagittis diam vitae hendrerit.

Vestibulum mattis varius massa eget tristique. Morbi dignissim ornare sapien.

Pellentesque ac vestibulum mi, vel consequat dui. Donec ut leo vel urna ullamcorper tincidunt in vel quam. Mauris finibus dictum tortor. Etiam dapibus arcu nunc, eget facilisis mi sodales vel. Phasellus arcu enim, pulvinar at maximus vel, efficitur vel nisi.

Proin porta pretium dui, in vulputate quam facilisis sed. Curabitur sollicitudin elit eget enim pretium, sed pharetra erat volutpat. Donec consectetur tincidunt condimentum. Cras libero sem, malesuada ac consequat vitae, faucibus vulputate ex. Duis aliquet ultricies lectus quis luctus. Vivamus vehicula nec dolor vel dictum. Praesent at lectus lacinia, accumsan ante eget, gravida eros.

Phasellus turpis arcu, aliquam ut tincidunt a, pretium et orci.

Phasellus quis nunc at quam dapibus bibendum ut non odio. Quisque hendrerit tempus velit, non bibendum nisl convallis quis. Nullam eu congue ipsum, non fermentum ante. Phasellus consectetur ipsum id libero congue elementum. Donec lacinia lobortis libero, id ultricies elit tempor in. Integer quis placerat metus, sed fringilla dui. Aenean lobortis lobortis dictum. Morbi quis neque arcu. Sed at rhoncus lacus. Sed pulvinar turpis quis mollis pretium.

Praesent ut lobortis orci. Etiam non orci ut felis egestas lacinia.

Suspendisse vitae porttitor ex. Sed non porta turpis, sodales elementum erat.

Pellentesque habitant morbi tristique senectus et netus et malesuada fames ac turpis egestas.

Class aptent taciti sociosqu ad litora torquent per conubia nostra, per inceptos himenaeos.

Vivamus feugiat tortor eget faucibus pellentesque. Morbi sit amet dui finibus sapien dapibus elementum vitae quis lacus. Cras faucibus quam non lectus vestibulum, quis tincidunt neque efficitur.

Praesent in tempus elit.

Sed vulputate blandit tellus et pulvinar.

Maecenas vel finibus risus. Pellentesque dolor dui, molestie eu leo vel, ultricies bibendum libero.

Nulla consequat malesuada accumsan.

Phasellus et urna vel erat pretium vestibulum ac lobortis arcu.

Quisque neque sapien, blandit sed nunc et, iaculis pretium dui. Aenean sed volutpat est, eu venenatis ante. Quisque sollicitudin ligula ornare tortor ornare blandit.

Proin ullamcorper non tortor sit amet convallis.

Quisque iaculis quis ex non finibus. Quisque hendrerit lacinia velit sodales accumsan. Sed vel mollis risus.

Suspendisse ornare risus lectus, vel eleifend lorem sagittis sed.

Cras tincidunt leo vel odio luctus, non fermentum turpis luctus.

Suspendisse feugiat elit sit amet lacus porttitor ornare.

Praesent eu luctus ex. Curabitur maximus bibendum nunc vitae semper.    % Раздел 1                                         % Глава 2
\chapter*{\centering ЗАКЛЮЧЕНИЕ}
\addcontentsline{toc}{chapter}{ЗАКЛЮЧЕНИЕ}    % Добавляем его в оглавление

В результате выполнения данной выпускной квалификационной работы было разработано программное обеспечение...

Таким образом, в процессе выполнения данной выпускной квалификационной работы были решены следующие задачи:
\begin{diplomaitemize}
    \item задача 1;
    \item задача 2;
    \item задача 3.
\end{diplomaitemize}

В пути дальнейшей работы входит...
Таким образом, для улучшения данной выпускной квалификационной работы в первую очередь необходимо выполнить ряд таких задач как:
\begin{diplomaitemize}
    \item задача 1;
    \item задача 2;
    \item задача 3.
\end{diplomaitemize}



Вода...                                                  % Заключение
\input{src/references.tex}                                                  % Библиографический список

\setcounter{totalchapter}{\value{chapter}} % Подсчёт количества глав

%%% Настройки для приложений
\appendix
% Оформление заголовков приложений ближе к ГОСТ:
%\setlength{\midchapskip}{20pt}
%\renewcommand*{\afterchapternum}{\par\nobreak\vskip \midchapskip}
%\renewcommand\thechapter{\Asbuk{chapter}} % Чтобы приложения русскими буквами нумеровались

%\include{include/dissertation/appendix}        % Приложения

%\setcounter{totalappendix}{\value{chapter}} % Подсчёт количества приложений

\end{document}