\section{Форматирование текста}\label{sec:chapter_1/section_1}

Мы можем сделать \textbf{жирный текст} и \textit{курсив}.

\section{Ссылки}\label{sec:chapter_1/section_2}

Сошлёмся на библиографию.
Одна ссылка: \cite{kuznetsov}.
Две ссылки: \cite{kuznetsov, creating_a_ros_package}.
Ещё ссылка: \cite{gost_9294_93}.

\section{Формулы}\label{sec:chapter_1/section_3}

Благодаря пакету \textit{icomma}, \LaTeX одинаково хорошо воспринимает в качестве десятичного разделителя и запятую (\(3,1415\)), и точку (\(3.1415\)).

\subsection{Ненумерованные одиночные формулы}\label{subsec:chapter_1/section_3/subsection_1}

Вот так может выглядеть формула, которую необходимо вставить в строку по тексту: \(x \approx \sin x\) при \(x \to 0\).

\subsection{Другая формула}\label{subsec:chapter_1/section_3/subsection_2}

Другой формулы не будет. Я не знаю математику в \LaTeX.

\subsection{Ещё формула}\label{subsec:chapter_1/section_3/subsection_3}

Ещё формулы не будет. Я не знаю математику в \LaTeX.

\section{Цитата}\label{sec:chapter_1/section_4}

Цитата: <<Lorem ipsum dolor sit amet>>.

\section{Перечисления}\label{sec:chapter_1/section_5}

Я добавил два новых стиля перечислений: diplomaenumerate и diplomaitemize.

\subsection{Числовое перечисление}\label{subsec:chapter_1/section_5/subsection_1}

Числовое перечисление:
\begin{diplomaenumerate}
    \item Первый пункт.
    \item Второй пункт.
    \item Третий пункт.
    \item Четвёртый пункт.
\end{diplomaenumerate}

\subsection{Перечисление пунктами}\label{subsec:chapter_1/section_5/subsection_2}

Перечисление пунктами:
\begin{diplomaitemize}
    \item первый пункт;
    \item второй пункт;
    \item третий пункт;
    \item четвёртый пункт.
\end{diplomaitemize}

\section{Тире и дефисы}\label{sec:chapter_1/section_6}

Стандартное \LaTeX тире --- это было тире.

Ещё есть короткое тире -- это было короткое тире.

Но нормоконтроль требует минус как тире - минус как тире.

\textbf{Тут подходит только последний вариант. Через такое - тире.}